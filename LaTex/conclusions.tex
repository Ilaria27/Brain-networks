\chapter{Conclusions}
\label{chap:4}

In this work we made an empirical summary of methods on \textit{Brain Network Classification}. We have seen how it is important to choose the right one for each dataset, together with the right parameters. In particular we have:

\begin{itemize}
	\item Described some basics arguments to go easier through the work;
	\item Made a summary of all methods we have found in the literature that make brain network classification, even if not all perfectly alligned with our experiments;
	\item Selected the methods to make experiments, so the ones that take as input wighted graphs in Adjacency matrices.
	\item Made experiments on those methods and comment on the results.
\end{itemize}

\section{Future works}

In future investigations we would like to:

\begin{itemize}
	\item Get better results, in terms of accuracy, from the methods we made experiments, investigating further their parameters;
	\item Find other methods in the literature that implemented brain network classification;
	\item Make experiments even on different methods, that take as input not only matrices;
	\item Some of the methods we have taken in account, also take in account the nodes of the graphs as real parts of our brain. For each method we could explore how good they do that, and consequently which are the relevant areas that influence the classification of particular mental desease. 
\end{itemize}