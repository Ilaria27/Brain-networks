\begin{abstract}
Our brain, as our entire body, is a perfect as complicated machine. For this reason is important to analyse it in an effective and efficient way. An important role in this purpose is given to machine learning and similar algorithmic methods. In fact, many papers have been written in this field on implemented methods of brain classification. In those papers the brain is represented as a graph or as a matrix and from that representation they want to recognize if a person is affected by a psychiatric disorder, such as Autism or Schizophrenia.  

In this work, the aim is to study and compare some of those methods. First of all is important to have in mind which are the basic elements with which the papers work, such as graphs, classification and brain connectomes. Then, the algorithms taken in consideration can be classified according to the technique used, like Neural Networks or features embeddings, so, some methods of different techniques are described. For each method many experiments are made, to see how they work, in which case are more useful, and how good are the results. The experiments are made with different datasets, to see how the methods adapt and how general they are. 

Future works can be implemented, studying more methods and maybe finding a method that could be easily used by neuroscientists. 
\end{abstract}